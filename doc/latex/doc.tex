%%%%%%%%%%%%%%%%%%%%%%%%%%%%%%%%%%%%%%%%%%%%%%%%%%%%%%%%%%%%%%%%%%%%%%%%%%%%%%%
%
%   This file is part of FLINT.
%
%   FLINT is free software; you can redistribute it and/or modify
%   it under the terms of the GNU General Public License as published by
%   the Free Software Foundation; either version 2 of the License, or
%   (at your option) any later version.
%
%   FLINT is distributed in the hope that it will be useful,
%   but WITHOUT ANY WARRANTY; without even the implied warranty of
%   MERCHANTABILITY or FITNESS FOR A PARTICULAR PURPOSE.  See the
%   GNU General Public License for more details.
%
%   You should have received a copy of the GNU General Public License
%   along with FLINT; if not, write to the Free Software
%   Foundation, Inc., 51 Franklin St, Fifth Floor, Boston, MA  02110-1301 USA
%
%%%%%%%%%%%%%%%%%%%%%%%%%%%%%%%%%%%%%%%%%%%%%%%%%%%%%%%%%%%%%%%%%%%%%%%%%%%%%%%
%%%%%%%%%%%%%%%%%%%%%%%%%%%%%%%%%%%%%%%%%%%%%%%%%%%%%%%%%%%%%%%%%%%%%%%%%%%%%%%
%
%   Copyright (C) 2007 William Hart, David Harvey
%   Copyright (C) 2010 Sebastian Pancratz
%
%%%%%%%%%%%%%%%%%%%%%%%%%%%%%%%%%%%%%%%%%%%%%%%%%%%%%%%%%%%%%%%%%%%%%%%%%%%%%%%

\documentclass[a4paper,10pt]{book}

%%%%%%%%%%%%
% geometry %
%%%%%%%%%%%%

\usepackage[hmargin=3.8cm,vmargin=3cm,a4paper,centering,twoside]{geometry}
\setlength{\headheight}{14pt}

% Dutch style of paragraph formatting, i.e. no indents
\setlength{\parskip}{1.3ex plus 0.2ex minus 0.2ex}
\setlength{\parindent}{0pt}

%%%%%%%%%%%%%%%%%%
% Other packages %
%%%%%%%%%%%%%%%%%%

\usepackage{amsmath,amsthm,amscd,amsfonts,amssymb}
\usepackage{cases}
\usepackage[all]{xy}

\usepackage{ifpdf}
\usepackage{paralist}
\usepackage{fancyhdr}
\usepackage{sectsty}
\usepackage{epigraph}
\usepackage{natbib}
\usepackage{url}
\usepackage[T1]{fontenc}
\usepackage{ae,aecompl}
\usepackage{booktabs}
\usepackage{multirow}
\usepackage{verbatim}
\usepackage{listings}

%%%%%%%%%%%%
% hyperref %
%%%%%%%%%%%%

\usepackage{hyperref}
\hypersetup{
    colorlinks=true,    % false: boxed links; true: colored links
    citecolor=green,    % color of links to bibliography
    filecolor=red,      % color of file links
    linkcolor=blue,     % color of internal links
    urlcolor=blue       % color of external links
}

\makeatletter
\newcommand\org@hypertarget{}
\let\org@hypertarget\hypertarget
\renewcommand\hypertarget[2]{%
    \Hy@raisedlink{\org@hypertarget{#1}{}}#2%
} 
\makeatother

\ifpdf
    \hypersetup{
        pdftitle={FLINT},
        pdfauthor={},
        pdfsubject={Computational mathematics},
        bookmarks=true,
        bookmarksnumbered=true,
        unicode=true,
        pdfstartview={FitH},
        pdfpagemode={UseOutlines}
    }
\fi

%%%%%%%%%%
% natbib %
%%%%%%%%%%

\bibpunct{[}{]}{,}{n}{}{}

\renewcommand{\bibname}{References}

%%%%%%%%%%%
% sectsty %
%%%%%%%%%%%

\allsectionsfont{\nohang\centering}

\sectionfont{\nohang\centering\large}

\makeatletter
\renewcommand{\@makechapterhead}[1]{%
\vspace*{50 pt}%
\begin{center}
\bfseries\Huge\S \thechapter.\ #1
\end{center}
\vspace*{40 pt}}
\makeatother

%%%%%%%%%%%%%%%%%%%%%
% Table of contents %
%%%%%%%%%%%%%%%%%%%%%

\usepackage{tocloft}

\addtolength{\cftsecnumwidth}{0.8em}
\addtolength{\cftbeforesecskip}{0.05em}

%%%%%%%%%%%%
% fancyhdr %
%%%%%%%%%%%%

\newcommand\nouppercase[1]{{%
    \let\uppercase\relax
    \let\MakeUppercase\relax
    \expandafter\let\csname MakeUppercase \endcsname\relax#1}%
}

\pagestyle{fancyplain}

\renewcommand{\chaptermark}[1]{\markboth{#1}{}}
\renewcommand{\sectionmark}[1]{\markright{\thesection\ #1}}
\fancyhf{}
\fancyhead[LE,RO]{\bfseries\thepage}
\fancyhead[LO]{\itshape\nouppercase{\rightmark}}
\fancyhead[RE]{\itshape\nouppercase{\leftmark}}
\renewcommand{\headrulewidth}{0pt}
\renewcommand{\footrulewidth}{0pt}

\fancypagestyle{plain}{%
  \fancyhead{}
  \renewcommand{\headrulewidth}{0pt}
}

\makeatletter
\def\cleardoublepage{\clearpage\if@twoside \ifodd\c@page\else
    \hbox{}
    \thispagestyle{plain}
    \newpage
    \if@twocolumn\hbox{}\newpage\fi\fi\fi}
\makeatother \clearpage{\pagestyle{plain}\cleardoublepage}

%%%%%%%
% url %
%%%%%%%

\makeatletter
\def\url@leostyle{%
  \@ifundefined{selectfont}{\def\UrlFont{\sf}}{\def\UrlFont{\small\ttfamily}}}
\makeatother
\urlstyle{leostyle}

%%%%%%%%%%%%%%%%
% Enumerations %
%%%%%%%%%%%%%%%%

\setlength{\pltopsep}{0.24em}
\setlength{\plpartopsep}{0em}
\setlength{\plitemsep}{0.24em}

% This should do what we want
%   \setdefaultenum{(i)}{(a)}{1.}{A}
% but it does not work for references, dropping the 
% parentheses.  The following hack does work.

\renewcommand{\theenumi}{(\roman{enumi})}
\renewcommand{\theenumii}{(\alph{enumii})}
\renewcommand{\theenumiii}{\arabic{enumiii}.}
\renewcommand{\theenumiv}{\Alph{enumiv}}

\renewcommand{\labelenumi}{\theenumi}
\renewcommand{\labelenumii}{\theenumii}
\renewcommand{\labelenumiii}{\theenumiii}
\renewcommand{\labelenumiv}{\theenumiv}

%%%%%%%%%%%%%%%%%%%%%%%%%
% Mathematical commands %
%%%%%%%%%%%%%%%%%%%%%%%%%

\renewcommand{\to}{\rightarrow}%         Right arrow
\newcommand{\into}{\hookrightarrow}%     Injection arrow
\newcommand{\onto}{\twoheadrightarrow}%  Surjection arrow

\providecommand{\abs}[1]{\lvert#1\rvert}%                  Absolute value
\providecommand{\absbig}[1]{\bigl\lvert#1\bigr\rvert}%     Absolute value
\providecommand{\absBig}[1]{\Bigl\lvert#1\Bigr\rvert}%     Absolute value
\providecommand{\absbigg}[1]{\biggl\lvert#1\biggr\rvert}%  Absolute value

\providecommand{\norm}[1]{\lVert#1\rVert}%               Norm
\providecommand{\normbig}[1]{\bigl\lVert#1\bigr\rVert}%  Norm
\providecommand{\normBig}[1]{\Bigl\lVert#1\Bigr\rVert}%  Norm

\providecommand{\floor}[1]{\left\lfloor#1\right\rfloor}%    Floor
\providecommand{\floorbig}[1]{\bigl\lfloor#1\bigr\rfloor}%  Floor
\providecommand{\floorBig}[1]{\Bigl\lfloor#1\Bigr\rfloor}%  Floor

\providecommand{\ceil}[1]{\left\lceil#1\right\rceil}%    Ceiling
\providecommand{\ceilbig}[1]{\bigl\lceil#1\bigr\rceil}%  Ceiling
\providecommand{\ceilBig}[1]{\Bigl\lceil#1\Bigr\rceil}%  Ceiling

\newcommand{\N}{\mathbf{N}}%  Natural numbers
\newcommand{\Z}{\mathbf{Z}}%  Integers
\newcommand{\Q}{\mathbf{Q}}%  Rationals

\DeclareMathOperator{\sgn}{sgn}

\allowdisplaybreaks[4]
%\numberwithin{equation}{section}

%%%%%%%%%%%%
% listings %
%%%%%%%%%%%%

\lstset{language=c}
\lstset{basicstyle=\ttfamily}
\lstset{breaklines=true}
\lstset{breakatwhitespace=true}
\lstset{keywordstyle=}
\lstset{morekeywords={mpz_t, mpq_t, mpz_poly_t, fmpz, fmpz_t, fmpz_poly_t}}
\lstset{escapechar=\%}

%%%%%%%%%%%%%%%%%%%%%%%%%%%
% FLINT specific commands %
%%%%%%%%%%%%%%%%%%%%%%%%%%%

\newcommand{\code}{\lstinline}

\DeclareMathOperator{\len}{len}

%%%%%%%%%%%%%%%%%%%%%%%%%%%%%%%%%%%%%%%%%%%%%%%%%%%%%%%%%%%%%%%%%%%%%%%%%%%%%%%
% DOCUMENT                                                                    %
%%%%%%%%%%%%%%%%%%%%%%%%%%%%%%%%%%%%%%%%%%%%%%%%%%%%%%%%%%%%%%%%%%%%%%%%%%%%%%%

\begin{document}

%%%%%%%%%%%%%%%%%%%%%%%%%%%%%%%%%%%%%%%%%%%%%%%%%%%%%%%%%%%%%%%%%%%%%%%%%%%%%%%
% FRONTMATTER                                                                 %
%%%%%%%%%%%%%%%%%%%%%%%%%%%%%%%%%%%%%%%%%%%%%%%%%%%%%%%%%%%%%%%%%%%%%%%%%%%%%%%

\frontmatter

\thispagestyle{plain}

\vfill
\hbox{%
\rule{1pt}{1.0\textheight}
\hspace*{0.05\textwidth}%
\parbox[b]{0.75\textwidth}{
\vbox{%
\vspace{0.1\textwidth}
{\noindent\Huge\bfseries FLINT}\\[2\baselineskip]
{\Large\itshape Fast Library for Number Theory}

\vspace{0.5\textheight}
{\large Version~2.0.0}\\[1.2\baselineskip]
{\large 32 January 2011}\\[1.2\baselineskip]
{\large *William Hart, **Fredrik Johansson, Sebastian Pancratz}\\[1.2\baselineskip]
{\large * Supported by EPSRC Grant EP/G004870/1}\\[1.2\baselineskip]
{\large ** Supported by Austrian Science Foundation (FWF) grant Y464-N18}\\[1.2\baselineskip]
}% end of vbox
}% end of parbox
}% end of hbox
\vfill


\clearpage

\tableofcontents

%%%%%%%%%%%%%%%%%%%%%%%%%%%%%%%%%%%%%%%%%%%%%%%%%%%%%%%%%%%%%%%%%%%%%%%%%%%%%%%
% MAINMATTER                                                                  %
%%%%%%%%%%%%%%%%%%%%%%%%%%%%%%%%%%%%%%%%%%%%%%%%%%%%%%%%%%%%%%%%%%%%%%%%%%%%%%%

\mainmatter

\chapter{Introduction}

FLINT is a C library of functions for doing number theory. It is highly 
optimised and can be compiled on numerous platforms.  FLINT also has the 
aim of providing support for multicore and multiprocessor computer 
architectures, though we do not yet provide this facility.

FLINT is currently maintained by William Hart of Warwick University in 
the UK.

As of version 1.1.0 FLINT supports 32 and 64 bit processors including 
x86, PPC, Alpha and Itanium processors, though in theory it compiles on any 
machine with GCC version 3.4 or later and with GMP version 4.2.1 or 
MPIR 0.9.0 or later.

FLINT is supplied as a set of modules, \code{fmpz}, \code{fmpz_poly}, etc., 
each of which can be linked to a C program making use of their functionality.

All of the functions in FLINT have a corresponding test function provided 
in an appropriately named test file.  For example, the function 
\code{fmpz_poly_add} located in \code{fmpz_poly/add.c} has test code in the 
file \code{fmpz_poly/test/t-add.c}.

\chapter{Building and using FLINT}

The easiest way to use FLINT is to build a shared library.  Simply download 
the FLINT tarball and untar it on your system.

FLINT requires GMP version 4.2.1 or later or MPIR version 0.9.0 or 
later (in GMP compatibility mode).  Set the environment variables 
\code{FLINT_GMP_LIB_DIR} and \code{FLINT_GMP_INCLUDE_DIR} to point 
to your GMP or MPIR library and include directories respectively. 
Alternatively you can set default values for these environment variables 
in the \code{flint_env} file.

Once the environment variables are set or defaults are set in 
\code{flint_env} simply type:
\begin{lstlisting}[language=bash]
source flint_env
\end{lstlisting}

in the main directory of the FLINT directory tree.  Finally type:
\begin{lstlisting}[language=bash]
make library
\end{lstlisting}

Move the library file \code{libflint.so}, \code{libflint.dll} or 
\code{libflint.dylib} (depending on your platform) into your library 
path and move all the \code{.h} files in the main directory of FLINT 
into your include path.

Now to use FLINT, simply include the appropriate header files for 
the FLINT modules you wish to use in your C program.  Then compile 
your program, linking against the FLINT library and GMP/MPIR with 
the options \code{-lflint -lgmp}.

\chapter{Test code}

Each module of FLINT has an extensive associated test module.  We 
strongly recommend running the test programs before relying on results 
from FLINT on your system. 

To make and run the test programs, simply type:
\begin{lstlisting}[language=bash]
make check
\end{lstlisting}

in the main FLINT directory.

\chapter{Reporting bugs}

The maintainer wishes to be made aware of any and all bugs.  Please send an 
email with your bug report to \url{hart_wb@yahoo.com}.

If possible please include details of your system, version of gcc, version 
of GMP/MPIR and precise details of how to replicate the bug.

Note that FLINT needs to be linked against version 4.2.1 or later of GMP 
or version 0.9.0 or later of MPIR (in GMP compatibility mode) and must be 
compiled with gcc version 3.4 or later.  In particular the compiler must be 
fully C99 compatible.

\chapter{Example programs}

FLINT comes with a number of example programs to demonstrate current and 
future FLINT features.  To build the example programs, type:

\begin{lstlisting}[language=bash]
make examples
\end{lstlisting}

The current example programs are:

\code{delta_qexp}  Computes the first $n$ terms of the delta function, e.g.\ 
\code{delta_qexp 1000000} will compute the first one million terms of the 
$q$-expansion of delta.

\code{BPTJCubes}  Implements the algorithm of Beck, Pine, Tarrant and Jensen 
for finding solutions to the equation $x^3+y^3+z^3 = k$.  This program 
outputs a file \code{output.log} containing parameters for reconstructing the 
first solution it finds, and then aborts.

\code{bernoulli_zmod} Compute many bernoulli numbers modulo a prime.  If no 
command line input is supplied it merely checks that the \code{bernoulli_zmod} 
function works for the first $2000$ primes.  If you specify an integer 
argument \code{n} on the command line, it computes the Bernoulli numbers 
$B_0, B_2, \dotsc, B_{p-1}$ modulo~$p$, where $p$ is the next prime from 
\code{n}.

\code{expmod}  Computes a very large modular exponentiation.  This is actually 
a basic pseudo primality test.

\code{Zmul}  Compares the output of the FLINT FFT with that of GMP for ever 
larger operands.

\code{thetaproduct}  Computes the congruent number theta function.  To run 
this you need to have openmp on your machine, you need a recent version of 
gcc (e.g. 4.3.x or 4.4.x) and you need to export \code{OMP_NUM_THREADS=16} 
or some factor of 16, depending on how many cores your machine has.  The 
code also expects a directory \code{storage} with \emph{plenty} of space 
where temporary files will be created.  Be warned that this code multiplies 
\emph{huge} integers which do not fit into memory and much disk space is 
used.  You also need a significant amount of memory on your machine, which 
must also be a 64 bit linux platform.  Parameters can be changed at the top 
of the file \code{thetaproduct.c}.  Primitive (squarefree) zeroes of the 
congruent number theta function curve will be computed up to 
\code{MOD * LIMIT} in the class $K$ modulo \code{MOD}.  At present 
\code{FILES1} and \code{FILES2} must be equal.  \code{LIMIT} must also be 
divisible by \code{BLOCK} and by \code{BUNDLE * FILES1}.  The code is not 
currently designed to correctly handle small problems. 

\chapter{FLINT macros}

The file \code{flint.h} contains various useful macros.

The macro constant \code{FLINT_BITS} is set at compile time to be the 
number of bits per limb on the machine.  FLINT requires it to be either 
32 or 64 bits.  Other architectures are not currently supported.

The macro constant \code{FLINT_D_BITS} is set at compile time to be the 
number of bits per double on the machine or the number of bits per limb, 
whichever is smaller.  This will have the value 53 or 32 on currently 
supported architectures.  Numerous functions using precomputed inverses 
only support operands up to \code{FLINT_D_BITS} bits, hence the macro.

The macro \code{FLINT_ABS(x)} returns the absolute value of \code{x}
for primitive signed numerical types.  It might fail for least negative 
values such as \code{INT_MIN} and \code{LONG_MIN}.

The macro \code{FLINT_MIN(x, y)} returns the minimum of \code{x} and 
\code{y} for primitive signed or unsigned numerical types.  This macro 
is only safe to use when \code{x} and \code{y} are of the same type, 
to avoid problems with integer promotion.

Similar to the previous macro, \code{FLINT_MAX(x, y)} returns the 
maximum of \code{x} and \code{y}.

The function \code{FLINT_BIT_COUNT(x)} returns the number of binary bits 
required to represent an \code{unsigned long x}.

%%%%%%%%%%%%%%%%%%%%%%%%%%%%%%%%%%%%%%%%%%%%%%%%%%%%%%%%%%%%%%%%%%%%%%%%%%%%%%%%
% Integers                                                                     %
%%%%%%%%%%%%%%%%%%%%%%%%%%%%%%%%%%%%%%%%%%%%%%%%%%%%%%%%%%%%%%%%%%%%%%%%%%%%%%%%

\chapter{fmpz}
\epigraph{Arbitrary precision integers}{}

\section{Introduction}

By default, an \code{fmpz_t} is implemented as an array of \code{fmpz}'s of 
length one to allow passing by reference as one can do with GMP/ MPIR's 
\code{mpz_t} type.  The \code{fmpz_t} type is simply a single limb, though 
the user does not need to be aware of this except in one specific case 
outlined below.

In all respects, \code{fmpz_t}'s act precisely like GMP/ MPIR's 
\code{mpz_t}'s, with automatic memory management, however, in the first 
place only one limb is used to implement them.  Once an \code{fmpz_t} 
overflows a limb then a multiprecision integer is automatically allocated 
and instead of storing the actual integer data the \code{long} which 
implements the type becomes an index into a FLINT wide array of 
\code{mpz_t}'s.

These internal implementation details are not important for the user to 
understand, except for three important things.

Firstly, \code{fmpz_t}'s will be more efficient than \code{mpz_t}'s for 
single limb operations, or more precisely for signed quantities whose 
absolute value does not exceed \code{FLINT_BITS - 2} bits.

Secondly, for small integers that fit into \code{FLINT_BITS - 2} bits 
much less memory will be used than for an \code{mpz_t}.  When very many 
\code{fmpz_t}'s are used, there can be important cache benefits on 
account of this.

Thirdly, it is important to understand how to deal with arrays of 
\code{fmpz_t}'s.  As for \code{mpz_t}'s, there is an underlying type, 
an \code{fmpz}, which can be used to create the array, e.g.\ 
\begin{lstlisting}
fmpz myarr[100];
\end{lstlisting}
Now recall that an \code{fmpz_t} is an array of length one of \code{fmpz}'s.
Thus, a pointer to an \code{fmpz} can be used in place of an \code{fmpz_t}.
For example, to find the sign of the third integer in our array we would 
write 
\begin{lstlisting}
int sign = fmpz_sgn(myarr + 2);
\end{lstlisting}

The \code{fmpz} module provides routines for memory management, basic 
manipulation and basic arithmetic.

Unless otherwise specified, all functions in this section permit aliasing 
between their input arguments and between their input and output 
arguments.

\section{Simple example}

The following example computes the square of the integer $7$ and prints 
the result.
\begin{lstlisting}[language=c]
#include "fmpz.h"
...
fmpz_t x, y;
fmpz_init(x);
fmpz_init(y);
fmpz_set_ui(x, 7);
fmpz_mul(y, x, x);
fmpz_print(x);
printf("^2 = ");
fmpz_print(y);
printf("\n");
fmpz_clear(x);
fmpz_clear(y);
\end{lstlisting}

The output is:
\begin{lstlisting}
7^2 = 49
\end{lstlisting}

We now describe the functions available in the \code{fmpz} module.

\input{fmpz.tex}

%%%%%%%%%%%%%%%%%%%%%%%%%%%%%%%%%%%%%%%%%%%%%%%%%%%%%%%%%%%%%%%%%%%%%%%%%%%%%%%%
% Integer vectors                                                              %
%%%%%%%%%%%%%%%%%%%%%%%%%%%%%%%%%%%%%%%%%%%%%%%%%%%%%%%%%%%%%%%%%%%%%%%%%%%%%%%%

\chapter{fmpz\_vec}
\epigraph{Vectors over $\Z$}{}

\input{fmpz_vec.tex}

%%%%%%%%%%%%%%%%%%%%%%%%%%%%%%%%%%%%%%%%%%%%%%%%%%%%%%%%%%%%%%%%%%%%%%%%%%%%%%%%
% Integer matrices                                                             %
%%%%%%%%%%%%%%%%%%%%%%%%%%%%%%%%%%%%%%%%%%%%%%%%%%%%%%%%%%%%%%%%%%%%%%%%%%%%%%%%

\chapter{fmpz\_mat}
\epigraph{Matrices over $\Z$}{}

\section{Introduction}

The \code{fmpz_mat_t} data type represents matrices of multiprecision integers.
The number of rows or columns in a matrix is permitted to be zero.

No automatic resizing is performed: in general, the user must provide
matrices of correct size for both input and output variables. Output
variables are \emph{not} allowed to be aliased with input variables unless
otherwise noted.

\input{fmpz_mat.tex}

%%%%%%%%%%%%%%%%%%%%%%%%%%%%%%%%%%%%%%%%%%%%%%%%%%%%%%%%%%%%%%%%%%%%%%%%%%%%%%%%
% Integer polynomials                                                          %
%%%%%%%%%%%%%%%%%%%%%%%%%%%%%%%%%%%%%%%%%%%%%%%%%%%%%%%%%%%%%%%%%%%%%%%%%%%%%%%%

\chapter{fmpz\_poly}
\epigraph{Polynomials over $\Z$}{}

\section{Introduction}

The \code{fmpz_poly_t} data type represents elements of $\Z[x]$. The 
\code{fmpz_poly} module provides routines for memory management, basic 
arithmetic, and conversions from or to other types.

Each coefficient of an \code{fmpz_poly_t} is an integer of the FLINT 
\code{fmpz_t} type.  There are two advantages of this model.  Firstly, 
the \code{fmpz_t} type is memory managed, so the user can manipulate 
individual coefficients of a polynomial without having to deal with 
tedious memory management.  Secondly, a coefficient of an 
\code{fmpz_poly_t} can be changed without changing the size of any 
of the other coefficients.

Unless otherwise specified, all functions in this section permit aliasing 
between their input arguments and between their input and output arguments.

\section{Simple example}

The following example computes the square of the polynomial $5x^3 - 1$.
\begin{lstlisting}[language=c]
#include "fmpz_poly.h"
...
fmpz_poly_t x, y;
fmpz_poly_init(x);
fmpz_poly_init(y);
fmpz_poly_set_coeff_ui(x, 3, 5);
fmpz_poly_set_coeff_si(x, 0, -1);
fmpz_poly_mul(y, x, x);
fmpz_poly_print(x); printf("\n");
fmpz_poly_print(y); printf("\n");
fmpz_poly_clear(x);
fmpz_poly_clear(y);
\end{lstlisting}

The output is:
\begin{lstlisting}
4  -1 0 0 5
7  1 0 0 -10 0 0 25
\end{lstlisting}

\section{Definition of the fmpz\_poly\_t type}

The \code{fmpz_poly_t} type is a typedef for an array of length~1 of 
\code{fmpz_poly_struct}'s.  This permits passing parameters of type 
\code{fmpz_poly_t} by reference in a manner similar to the way GMP integers 
of type \code{mpz_t} can be passed by reference. 

In reality one never deals directly with the \code{struct} and simply deals 
with objects of type \code{fmpz_poly_t}.  For simplicity we will think of an 
\code{fmpz_poly_t} as a \code{struct}, though in practice to access fields 
of this \code{struct}, one needs to dereference first, e.g.\ to access the 
\code{length} field of an \code{fmpz_poly_t} called \code{poly1} one writes 
\code{poly1->length}. 

An \code{fmpz_poly_t} is said to be \emph{normalised} if either 
\code{length} is zero, or if the leading coefficient of the polynomial is 
non-zero.  All \code{fmpz_poly} functions expect their inputs to be 
normalised, and unless otherwise specified they produce output that is 
normalised. 

It is recommended that users do not access the fields of an 
\code{fmpz_poly_t} or its coefficient data directly, but make use of the 
functions designed for this purpose, detailed below.

Functions in \code{fmpz_poly} do all the memory management for the user. 
One does not need to specify the maximum length or number of limbs per 
coefficient in advance before using a polynomial object.  FLINT reallocates 
space automatically as the computation proceeds, if more space is required. 
Each coefficient is also managed separately, being resized as needed, 
independently of the other coefficients.

We now describe the functions available in \code{fmpz_poly}.

\input{fmpz_poly.tex}

%%%%%%%%%%%%%%%%%%%%%%%%%%%%%%%%%%%%%%%%%%%%%%%%%%%%%%%%%%%%%%%%%%%%%%%%%%%%%%%%
% Rational polynomials                                                         %
%%%%%%%%%%%%%%%%%%%%%%%%%%%%%%%%%%%%%%%%%%%%%%%%%%%%%%%%%%%%%%%%%%%%%%%%%%%%%%%%

\chapter{fmpq\_poly}
\epigraph{Polynomials over $\Q$}{}

\section{Introduction}

The \code{fmpq_poly_t} data type represents elements of $\Q[x]$. The 
\code{fmpq_poly} module provides routines for memory management, basic 
arithmetic, and conversions from or to other types.

A rational polynomial is stored as the quotient of an integer polynomial 
and an integer denominator.  To be more precise, the coefficient vector 
of the numerator can be accessed with the function \code{fmpq_poly_numref()} 
and the denominator with \code{fmpq_poly_denref()}.  Although one can 
construct use cases in which a representation as a list of rational 
coefficients would be beneficial, the choice made here is typically 
more efficient.

We can obtain a unique representation based on this choice by enforcing, 
for non-zero polynomials, that the numerator and denominator are coprime 
and that the denominator is positive.  The unique representation of the 
zero polynomial is chosen as $0/1$.

Similar to the situation in the \code{fmpz_poly_t} case, an 
\code{fmpq_poly_t} object also has a \code{length} parameter, which 
denotes the length of the vector of coefficients of the numerator. 
We say a polynomial is \emph{normalised} either if this length is zero 
or if the leading coefficient is non-zero.

We say a polynomial is in \emph{canonical} form if it is given in the 
unique representation discussed above and normalised.

The functions provided in this module roughly fall into two categories:

On the one hand, there are functions mainly provided for the user, whose 
names do not begin with an underscore.  These typically operate on 
polynomials of type \code{fmpq_poly_t} in canonical form and, unless 
specified otherwise, permit aliasing between their input arguments and 
between their output arguments.

On the other hand, there are versions of these functions whose names are 
prefixed with a single underscore.  These typically operate on polynomials 
given in the form of a triple of object of types \code{fmpz *}, 
\code{fmpz_t}, and \code{long}, containing the numerator, denominator and 
length, respectively.  In general, these functions expect their input to 
be normalised, i.e.\ they do not allow zero padding, and to be in lowest 
terms, and they do not allow their input and output arguments to be aliased.

\input{fmpq_poly.tex}

%%%%%%%%%%%%%%%%%%%%%%%%%%%%%%%%%%%%%%%%%%%%%%%%%%%%%%%%%%%%%%%%%%%%%%%%%%%%%%%%
% Single limb arithmetic                                                       %
%%%%%%%%%%%%%%%%%%%%%%%%%%%%%%%%%%%%%%%%%%%%%%%%%%%%%%%%%%%%%%%%%%%%%%%%%%%%%%%%

\chapter{ulong\_extras}
\epigraph{Unsigned single limb arithmetic}{}

\input{ulong_extras.tex}

%%%%%%%%%%%%%%%%%%%%%%%%%%%%%%%%%%%%%%%%%%%%%%%%%%%%%%%%%%%%%%%%%%%%%%%%%%%%%%%%
%                                                                              %
%%%%%%%%%%%%%%%%%%%%%%%%%%%%%%%%%%%%%%%%%%%%%%%%%%%%%%%%%%%%%%%%%%%%%%%%%%%%%%%%

\chapter{nmod\_vec}
\epigraph{Vectors over $\Z / n \Z$ for word-sized moduli}{}

\input{nmod_vec.tex}

%%%%%%%%%%%%%%%%%%%%%%%%%%%%%%%%%%%%%%%%%%%%%%%%%%%%%%%%%%%%%%%%%%%%%%%%%%%%%%%%
%                                                                              %
%%%%%%%%%%%%%%%%%%%%%%%%%%%%%%%%%%%%%%%%%%%%%%%%%%%%%%%%%%%%%%%%%%%%%%%%%%%%%%%%

\chapter{nmod\_poly}
\epigraph{Polynomials over $\Z / n \Z$ for word-sized moduli}{}

\input{nmod_poly.tex}

%%%%%%%%%%%%%%%%%%%%%%%%%%%%%%%%%%%%%%%%%%%%%%%%%%%%%%%%%%%%%%%%%%%%%%%%%%%%%%%%
% Matrices over Z / nZ for word-sized moduli                                   %
%%%%%%%%%%%%%%%%%%%%%%%%%%%%%%%%%%%%%%%%%%%%%%%%%%%%%%%%%%%%%%%%%%%%%%%%%%%%%%%%

\chapter{nmod\_mat}
\epigraph{Matrices over $\Z / n \Z$ for word-sized moduli}{}

\section{Introduction}

An \code{nmod_mat_t} represents a matrix of integers modulo $n$, for any 
non-zero $n$ that fits in a single limb, up to $2^{32}-1$ or $2^{64}-1$. The
implementation uses functions and data types of the \code{nmod_vec} module
for low-level operations.

One or both dimensions of a matrix may be zero.

Except where otherwise noted, it is assumed all entries in input
data are already reduced in the range $[0, n)$. It is also assumed that
all arguments have the same modulus.

Functions that require the modulus to be a prime number document this
requirement explicitly.

\input{nmod_mat.tex}

%%%%%%%%%%%%%%%%%%%%%%%%%%%%%%%%%%%%%%%%%%%%%%%%%%%%%%%%%%%%%%%%%%%%%%%%%%%%%%%%
% Arithmetic functions                                                         %
%%%%%%%%%%%%%%%%%%%%%%%%%%%%%%%%%%%%%%%%%%%%%%%%%%%%%%%%%%%%%%%%%%%%%%%%%%%%%%%%

\chapter{arith}
\epigraph{Arithmetic functions}{}

\section{Introduction}

This module implements arithmetic functions, number-theoretic and 
combinatorial special number sequences and polynomials.

\input{arith.tex}

%%%%%%%%%%%%%%%%%%%%%%%%%%%%%%%%%%%%%%%%%%%%%%%%%%%%%%%%%%%%%%%%%%%%%%%%%%%%%%%%
% LLL                                                                          %
%%%%%%%%%%%%%%%%%%%%%%%%%%%%%%%%%%%%%%%%%%%%%%%%%%%%%%%%%%%%%%%%%%%%%%%%%%%%%%%%

\chapter{LLL}
\epigraph{Functionality supporting the Lenstra--Lenstra--Lov\'asz algorithm 
          for lattice reduction}{}

\input{LLL.tex}

%%%%%%%%%%%%%%%%%%%%%%%%%%%%%%%%%%%%%%%%%%%%%%%%%%%%%%%%%%%%%%%%%%%%%%%%%%%%%%%%
% mpn_extras                                                                   %
%%%%%%%%%%%%%%%%%%%%%%%%%%%%%%%%%%%%%%%%%%%%%%%%%%%%%%%%%%%%%%%%%%%%%%%%%%%%%%%%

\chapter{mpn\_extras}

\input{mpn_extras.tex}

%%%%%%%%%%%%%%%%%%%%%%%%%%%%%%%%%%%%%%%%%%%%%%%%%%%%%%%%%%%%%%%%%%%%%%%%%%%%%%%
% BACKMATTER                                                                  %
%%%%%%%%%%%%%%%%%%%%%%%%%%%%%%%%%%%%%%%%%%%%%%%%%%%%%%%%%%%%%%%%%%%%%%%%%%%%%%%

\backmatter

\phantomsection
\addcontentsline{toc}{chapter}{References}
\bibliographystyle{amsplain}
\bibliography{flint}

\end{document}
